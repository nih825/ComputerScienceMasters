\documentclass[11pt]{article}
\usepackage{ulem}
\usepackage{graphicx}
\usepackage{epstopdf}
\usepackage{amssymb}
\usepackage{amsmath}
\usepackage{listings}
\usepackage{color}
\usepackage{hyperref}
 
\definecolor{dkgreen}{rgb}{0,0.6,0}
\definecolor{gray}{rgb}{0.5,0.5,0.5}
\definecolor{mauve}{rgb}{0.58,0,0.82}
 
\lstset{ %
  language=C,                % the language of the code
  basicstyle=\footnotesize,           % the size of the fonts that are used for the code
  numbers=left,                   % where to put the line-numbers
  numberstyle=\tiny\color{gray},  % the style that is used for the line-numbers
  stepnumber=1,                   % the step between two line-numbers. If it's 1, each line 
                                  % will be numbered
  numbersep=5pt,                  % how far the line-numbers are from the code
  backgroundcolor=\color{white},      % choose the background color. You must add \usepackage{color}
  showspaces=false,               % show spaces adding particular underscores
  showstringspaces=false,         % underline spaces within strings
  showtabs=false,                 % show tabs within strings adding particular underscores
  frame=single,         
           % adds a frame around the code
  rulecolor=\color{black},        % if not set, the frame-color may be changed on line-breaks within not-black text (e.g. commens (green here))
  tabsize=2,                      % sets default tabsize to 2 spaces
  captionpos=b,                   % sets the caption-position to bottom
  breaklines=true,                % sets automatic line breaking
  breakatwhitespace=false,        % sets if automatic breaks should only happen at whitespace
  title=\lstname,                   % show the filename of files included with \lstinputlisting;
                                  % also try caption instead of title
  keywordstyle=\color{blue},          % keyword style
  commentstyle=\color{dkgreen},       % comment style
  stringstyle=\color{mauve},         % string literal style
  escapeinside={\%*}{*)},            % if you want to add a comment within your code
  morekeywords={*,...}               % if you want to add more keywords to the set
}

\begin{document}
\title{Lesson 3 outline} 
\author{CSPP58001: Numerical Methods \\ TA: Kyle Gerard Felker}
\date{\today}
\maketitle

\section{Returning to the heat equation...}
In the beginning of this lecture, we introduce another scheme for modeling the heat equation--- the Backwards Euler (BECS) a.k.a. backwards difference method. Refer to $\texttt{lesson2.pdf}$ for the mathematics for the 1D system. The take away point is that this method is \emph{implicit} instead of explicit. That is to say, you have to solve a linear system to get your answer because the temperature at any time $t$ is written as a function of the temperature at $t+1$. The PDF also introduces the Crank-Nicholson scheme, which combines features of backwards and forwards difference methods. 

\subsection{Going 2D}
We write the analytic heat equation in the form of a 2D problem:

$$ \frac{\partial T}{\partial t} = \alpha [ \frac{\partial^2 T}{\partial x\partial x} + \frac{\partial^2 T}{\partial y\partial y} ] = \nabla^2T$$
where the $\nabla^2$ is referred to as the Laplacian operator.

Now, using FTCS scheme, we have a larger explicit formula
$$ T_{i,j}^{n+1} = T_{i,j}^n + \frac{\alpha \delta t}{\delta x^2} [ T_{i+1,j}^n +  T_{i-1,j}^n +  T_{i,j+1}^n  +  T_{i,j-1}^n - 4 T_{i,j}^n ] $$  

For BECS,
$$ T_{i,j}^{n+1} = T_{i,j}^n + \frac{\alpha \delta t}{\delta x^2} [ T_{i+1,j}^{n+1} +  T_{i-1,j}^{n+1} +  T_{i,j+1}^{n+1}  +  T_{i,j-1}^{n+1} - 4 T_{i,j}^{n+1} ] $$
The majority of this lecture consisted of prototyping these schemes in MATLAB. The codes are available on the Dropbox.
\end{document}