\documentclass[11pt]{article}
\usepackage{ulem}
\usepackage{graphicx}
\usepackage{epstopdf}
\usepackage{amssymb}
\usepackage{amsmath}
\usepackage{listings}
\usepackage{color}
\usepackage{hyperref}
 
\definecolor{dkgreen}{rgb}{0,0.6,0}
\definecolor{gray}{rgb}{0.5,0.5,0.5}
\definecolor{mauve}{rgb}{0.58,0,0.82}
 
\lstset{ %
  language=C,                % the language of the code
  basicstyle=\footnotesize,           % the size of the fonts that are used for the code
  numbers=left,                   % where to put the line-numbers
  numberstyle=\tiny\color{gray},  % the style that is used for the line-numbers
  stepnumber=1,                   % the step between two line-numbers. If it's 1, each line 
                                  % will be numbered
  numbersep=5pt,                  % how far the line-numbers are from the code
  backgroundcolor=\color{white},      % choose the background color. You must add \usepackage{color}
  showspaces=false,               % show spaces adding particular underscores
  showstringspaces=false,         % underline spaces within strings
  showtabs=false,                 % show tabs within strings adding particular underscores
  frame=single,         
           % adds a frame around the code
  rulecolor=\color{black},        % if not set, the frame-color may be changed on line-breaks within not-black text (e.g. commens (green here))
  tabsize=2,                      % sets default tabsize to 2 spaces
  captionpos=b,                   % sets the caption-position to bottom
  breaklines=true,                % sets automatic line breaking
  breakatwhitespace=false,        % sets if automatic breaks should only happen at whitespace
  title=\lstname,                   % show the filename of files included with \lstinputlisting;
                                  % also try caption instead of title
  keywordstyle=\color{blue},          % keyword style
  commentstyle=\color{dkgreen},       % comment style
  stringstyle=\color{mauve},         % string literal style
  escapeinside={\%*}{*)},            % if you want to add a comment within your code
  morekeywords={*,...}               % if you want to add more keywords to the set
}

\begin{document}
\title{Lesson 2 outline} 
\author{CSPP58001: Numerical Methods \\ TA: Kyle Gerard Felker}
\date{\today}
\maketitle

\section{Homework clarifications}

\begin{itemize}
\item \#1 A reminder, we want the matrix \emph{that transforms your original matrix} upon multiplication to row echelon form, not the Gaussian eliminated matrix.

\item \#9 -- we do not require that you time the largest possible matrix on your system, rather we want a range of matrix sizes --- the largest problem should take many seconds to a minute to Gaussian eliminate. 

\item \#10 Proving these properties should be taken in a loose sense; we do not require a rigourous mathematics class proof. Feel free to show the properties are true for a general 3x3 matrix. 

\end{itemize}
\section{Back to Gaussian elimination...}
What can go wrong with the naive Gaussian elimination algorithm?

\begin{itemize}
\item a pivot element could be zero $\rightarrow$ algorithm divides by zero and yields nonsense results

\item ``small'' pivots can yield significant roundoff error on computers with finite precision variables
\end{itemize}
We have three strategies to contend with these algorithmic issues:

\begin{enumerate}
\item (Partial pivoting) Swap rows with row that contains largest absolute value pivot element among rows below pivot position

\item (Scaled pivoting) Save as above but interpret ``largest" as largest relative to other elements in the row

\item (Full pivoting) Look for largest pivot in any entry below current row and to the right of the current column. Swap rows and columns as necessary, while tracking the column changes in the solution vector.

\end{enumerate}
All strategies are heuristical, and they entail different computational costs. 

\section{Floating point representation}
See $\texttt{lesson2.pdf} $ and corresponding PowerPoint in the Dropbox to see the lecture contents regarding the use of floating point representation and its consequences for our numerical algorithms. 

\section{Numerical methods for PDEs --- the heat equation}
See  $\texttt{lesson2.pdf} $


\end{document}