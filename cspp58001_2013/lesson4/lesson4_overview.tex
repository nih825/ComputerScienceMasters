\documentclass[11pt]{article}
\usepackage{ulem}
\usepackage{graphicx}
\usepackage{epstopdf}
\usepackage{amssymb}
\usepackage{amsmath}
\usepackage{listings}
\usepackage{color}
\usepackage{hyperref}
 
\definecolor{dkgreen}{rgb}{0,0.6,0}
\definecolor{gray}{rgb}{0.5,0.5,0.5}
\definecolor{mauve}{rgb}{0.58,0,0.82}
 
\lstset{ %
  language=C,                % the language of the code
  basicstyle=\footnotesize,           % the size of the fonts that are used for the code
  numbers=left,                   % where to put the line-numbers
  numberstyle=\tiny\color{gray},  % the style that is used for the line-numbers
  stepnumber=1,                   % the step between two line-numbers. If it's 1, each line 
                                  % will be numbered
  numbersep=5pt,                  % how far the line-numbers are from the code
  backgroundcolor=\color{white},      % choose the background color. You must add \usepackage{color}
  showspaces=false,               % show spaces adding particular underscores
  showstringspaces=false,         % underline spaces within strings
  showtabs=false,                 % show tabs within strings adding particular underscores
  frame=single,         
           % adds a frame around the code
  rulecolor=\color{black},        % if not set, the frame-color may be changed on line-breaks within not-black text (e.g. commens (green here))
  tabsize=2,                      % sets default tabsize to 2 spaces
  captionpos=b,                   % sets the caption-position to bottom
  breaklines=true,                % sets automatic line breaking
  breakatwhitespace=false,        % sets if automatic breaks should only happen at whitespace
  title=\lstname,                   % show the filename of files included with \lstinputlisting;
                                  % also try caption instead of title
  keywordstyle=\color{blue},          % keyword style
  commentstyle=\color{dkgreen},       % comment style
  stringstyle=\color{mauve},         % string literal style
  escapeinside={\%*}{*)},            % if you want to add a comment within your code
  morekeywords={*,...}               % if you want to add more keywords to the set
}

\begin{document}
\title{Lesson 4 outline} 
\author{CSPP58001: Numerical Methods \\ TA: Kyle Gerard Felker}
\date{\today}
\maketitle

\section{Cubic splines}
We take a brief detour to describe another application of matrix computations. A very common problem in data analysis involves interpolation between discretely sampled measurements. If you want to know what the observation should be at a point between your samples, you need to interpolate to get a reasonable guess.

There are many ways to do this. You could come up with a single polynomial that best describes the distribution of points, but polynomial interpolation can have disadvantages with high oscillations. 

One popular method is spline interpolation. Instead of using a single polynomial for the entire range, we come up with a piecewise polynomial composed of separate polynomials between each two data points subject to some restrictions.

The simplest form of spline interpolation would be to draw straight lines between the data points. This is known as linear interpolation. 

We are going to discuss a popular variant--- cubic splines. Say you have $N+1$ observations $f_i$ at $N+1$ different independent variables $x_i$. Our goal is to find $N$ cubic polynomials $S_0,S_1,\ldots,S_{N-1}$ that satisfy

\begin{enumerate}

\item $S_i(x_i) = f_i$ (interpolation) $i \in [0,N-1]$
\item $S_i(x_{i+1}) = S_{i+1}(x_{i+1})$ (continuity) $i \in [0,N-2]$
\item  $S'_i(x_{i+1}) = S'_{i+1}(x_{i+1})$ (continuity of first derivative) $i \in [0,N-2]$
\item  $S''_i(x_{i+1}) = S''_{i+1}(x_{i+1})$ (continuity of second derivative) $i \in [0,N-2]$

\end{enumerate}
This gives you $(N+1) + 3(N-1) = 4N -2$ independent constraints. For $N$ cubic polynomials of the form 
$$ S_i(x) = a_ix^3 + b_ix^2 +c_ix +d_i $$
we have $4N$ free variables in the form of the coefficients. We lack 2 constraints. The remaining constraints come from the boundary conditions. The most common boundary conditions are ``natural'' boundary conditions.
 
It is clear that with these constraints, the piecewise polynomials coefficients depend on the coefficients of their neighboring polynomials. 

You will have to make a cubic spline interpolator on HW3. 

\section{Understanding stability constraints}
See $\texttt{lesson4.pdf}$.


\end{document}